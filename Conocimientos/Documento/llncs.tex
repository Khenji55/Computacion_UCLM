% This is LLNCS.DOC the documentation file of
% the LaTeX2e class from Springer-Verlag
% for Lecture Notes in Computer Science, version 2.4
\documentclass{llncs}
\usepackage{llncsdoc}
%
\begin{document}
\markboth{\LaTeXe{} Class for Lecture Notes in Computer
Science}{\LaTeXe{} Class for Lecture Notes in Computer Science}
\thispagestyle{empty}
\begin{flushleft}
\LARGE\bfseries Instructions for Authors\\
Coding with \LaTeX\\[2cm]
\end{flushleft}
\rule{\textwidth}{1pt}
\vspace{2pt}
\begin{flushright}
\Huge
\begin{tabular}{@{}l}
\LaTeXe{} Class\\
for Lecture Notes\\
in Computer Science\\[6pt]
{\Large Version 2.4}
\end{tabular}
\end{flushright}
\rule{\textwidth}{1pt}
\vfill
%\begin{flushleft}
%\large\itshape
%\begin{tabular}{@{}l}
%{\Large\upshape\bfseries Springer}\\[8pt]
%Berlin\enspace Heidelberg\enspace New\kern0.1em York\\[5pt]
%Barcelona\enspace Budapest\enspace Hong\kern0.2em Kong\\[5pt]
%London\enspace Milan\enspace Paris\enspace\\[5pt]
%Santa\kern0.2em Clara\enspace Singapore\enspace Tokyo
%\end{tabular}
%\end{flushleft}
\newpage
%
\section*{For further information please contact us:}
%
\begin{flushleft}
\begin{tabular}{l@{\quad}l@{\hspace{3mm}}l@{\qquad}l}
$\bullet$&\multicolumn{3}{@{}l}{\bfseries LNCS Editorial Office}\\[1mm]
&\multicolumn{3}{@{}l}{Springer-Verlag}\\
&\multicolumn{3}{@{}l}{Computer Science Editorial}\\
&\multicolumn{3}{@{}l}{Tiergartenstra�e 17}\\
&\multicolumn{3}{@{}l}{69121 Heidelberg}\\
&\multicolumn{3}{@{}l}{Germany}\\[0.5mm]
 & Tel:       & +49-6221-487-8706\\
 & Fax:       & +49-6221-487-8588\\
 & e-mail:    & \tt lncs@springer.com    & for editorial questions\\
 &            & \tt texhelp@springer.de & for \TeX{} problems\\[2mm]
\noalign{\rule{\textwidth}{1pt}}
\noalign{\vskip2mm}
%
%{\tt svserv@vax.ntp.springer.de}\hfil first try the \verb|help|
%command.
%
$\bullet$&\multicolumn{3}{@{}l}{\bfseries We are also reachable through the world wide web:}\\[1mm]
         &\multicolumn{2}{@{}l}{\texttt{http://www.springer.com}}&Springer Global Website\\
         &\multicolumn{2}{@{}l}{\texttt{http://www.springer.com/lncs}}&LNCS home page\\
         &\multicolumn{2}{@{}l}{\texttt{http://www.springerlink.com}}&data repository\\
         &\multicolumn{2}{@{}l}{\texttt{ftp://ftp.springer.de}}&FTP server
\end{tabular}
\end{flushleft}


%
\newpage
\tableofcontents
\newpage
%
\section{Introduction}
%
Authors wishing to code their contribution
with \LaTeX{}, as well as those who have already coded with \LaTeX{},
will be provided with a document class that will give the text the
desired layout. Authors are requested to
adhere strictly to these instructions; {\em the class
file must not be changed}.

The text output area is automatically set within an area of
12.2\,cm horizontally  and 19.3\,cm vertically.

If you are already familiar with \LaTeX{}, then the
LLNCS class should not give you any major difficulties.
It will change the layout to the required LLNCS style
(it will for instance define the layout of \verb|\section|).
We had to invent some extra commands,
which are not provided by \LaTeX{} (e.g.\
\verb|\institute|, see also Sect.\,\ref{contbegin})

For the main body of the paper (the text) you
should use the commands of the standard \LaTeX{} ``article'' class.
Even if you are familiar with those commands, we urge you to read
this entire documentation thoroughly. It contains many suggestions on
how to use our commands properly; thus your paper
will be formatted exactly to LLNCS standard.
For the input of the references at the end of your contribution,
please follow our instructions given in Sect.\,\ref{refer} References.

The majority of these hints are not specific for LLNCS; they may improve
your use of \LaTeX{} in general.
Furthermore, the documentation provides suggestions about the proper
editing and use
of the input files (capitalization, abbreviation etc.) (see
Sect.\,\ref{refedit} How to Edit Your Input File).%
\end{document}
