% This is LLNCS.DOC the documentation file of
% the LaTeX2e class from Springer-Verlag
% for Lecture Notes in Computer Science, version 2.4
\documentclass{llncs}
\usepackage{llncsdoc}
\usepackage[latin1]{inputenc}
%
\begin{document}
\thispagestyle{empty}
\begin{flushleft}
\LARGE\bfseries Anteproyecto\\
\end{flushleft}
\rule{\textwidth}{1pt}
\vspace{2pt}
\begin{flushright}
\Huge
\begin{tabular}{@{}l}
Sistema Experto de\\
recomendaci�n de\\
Lenguaje de Programaci�n\\[6pt]
\end{tabular}
\end{flushright}
\rule{\textwidth}{1pt}
\vfill
%\begin{flushleft}
%\large\itshape
%\begin{tabular}{@{}l}
%{\Large\upshape\bfseries Springer}\\[8pt]
%Berlin\enspace Heidelberg\enspace New\kern0.1em York\\[5pt]
%Barcelona\enspace Budapest\enspace Hong\kern0.2em Kong\\[5pt]
%London\enspace Milan\enspace Paris\enspace\\[5pt]
%Santa\kern0.2em Clara\enspace Singapore\enspace Tokyo
%\end{tabular}
%\end{flushleft}
\newpage
%
\tableofcontents
\newpage
%
\section{Autores}

El proyecto ser� llevado a cabo por Juan Carlos Fern�ndez Dur�n y Victor Gualdras de la Cruz, alumnos de Sistemas basados en el Conocimiento.

\section{Objetivos y Alcance}

El objetivo de nuestro sistema experto es proporcionar al usuario un lenguaje de programaci�n que se adapte
a sus problemas y necesidades, pudiendo estas diferir de varias formas, los campos de aplicaci�n para nuestro
sistema experto ser�n:

\begin{itemize}
  \item Aplicaciones de sobremesa
  \item Aplicaciones multiplataforma
  \item P�ginas Web
  \item Aprendizaje
  \item Sistemas Expertos
  \item Videojuegos
  \item Aplicaciones para m�viles
  \item Microcontroladores
  \item Conocer un determinado paradigma (POO por ejemplo)
\end{itemize}

\section{Los Expertos}

El conocimiento ser� adquirido principalmente de 3 expertos, siendo todos profesores con actividades docentes
en la Escuela Superior de Inform�tica de la UCLM

\begin{itemize}
  \item \textbf{Pascual Juli�n Iranzo}: Doctorado en Ciencias de la Computaci�n, con actividad docente en
    L�gica y Programaci�n declarativa durante 18 a�os, experto en Sistemas Inteligentes proporcionar� el
    conocimiento necesario acerca de la rama de la Programaci�n Declarativa, sus ventajas frente a la
    Programaci�n Imperativa as� como de Sistemas Inteligentes.
  \item \textbf{David Villa Alises}: Doctorado en Ingener�a Inform�tica, miembro del grupo de investigaci�n
    \textbf{Arco} con actividad docente en Redes de Computadores y Sistemas Distribuidos.
  \item \textbf{Ismael Caballero Mu�oz-Reja}:Experiencia profesional en el mundo de la inform�tica, proporcionar� la visi�n m�s empresarial acerca de los lenguajes de programaci�n, actualmente desempe�a su actividad docente en la Escuela Superior de Inform�tica de la UCLM en asignaturas como Ingenier�a del Software II
\end{itemize}


\end{document}

% \\   Forzar algo m�s de espacio entre p�rrafos

%  \textbf{Texto en negrita}

%  \begin{itemize}
%    \item Tu lista
%    \item de objetos
%  \end{itemize} Adem�s se puede anidar con otro begin itemize para sublistas de objetos

% \newpage  Forzar nueva p�gina

% Las im�genes a�n no las entiendo porque es complicado, lo suyo es echarle un google

% Cualquier otra cosa que se necesite echandole 1 google en ingl�s siempre hay alguien que lo resuelve,
% est� muy bien documentado, yo ya te digo que no he necesitado aprender LaTeX como tal si no que me he
% puesto sobre la marcha buscando lo que necesitaba y aqu� estoy
