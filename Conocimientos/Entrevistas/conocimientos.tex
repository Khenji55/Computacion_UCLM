% This is LLNCS.DOC the documentation file of
% the LaTeX2e class from Springer-Verlag
% for Lecture Notes in Computer Science, version 2.4
\documentclass{llncs}
\usepackage[spanish]{babel}
\usepackage{llncsdoc}
\usepackage[latin1]{inputenc}
\usepackage{graphicx}
%
\begin{document}
\thispagestyle{empty}
\rule{\textwidth}{1pt}
\vspace{2pt}
\begin{flushright}
\Huge
\begin{tabular}{@{}l}
Sistema Experto de\\
recomendaci�n de\\
lenguaje de programaci�n\\[6pt]
\end{tabular}
\end{flushright}
\rule{\textwidth}{1pt}
\begin{flushleft}
\LARGE\bfseries Victor Gualdras de la Cruz\\
Juan Carlos Fern�ndez Dur�n\\[2cm]
\end{flushleft}
\vfill

\addto\captionsspanish{%
  \renewcommand\contentsname{}}
%\begin{flushleft}
%\large\itshape
%\begin{tabular}{@{}l}
%{\Large\upshape\bfseries Springer}\\[8pt]
%Berlin\enspace Heidelberg\enspace New\kern0.1em York\\[5pt]
%Barcelona\enspace Budapest\enspace Hong\kern0.2em Kong\\[5pt]
%London\enspace Milan\enspace Paris\enspace\\[5pt]
%Santa\kern0.2em Clara\enspace Singapore\enspace Tokyo
%\end{tabular}
%\end{flushleft}
\newpage
%

\tableofcontents
\newpage
%

\section{Entrevista}


\end{document}

% \\   Forzar algo m�s de espacio entre p�rrafos

%  \textbf{Texto en negrita}

%  \begin{itemize}
%    \item Tu lista
%    \item de objetos
%  \end{itemize} Adem�s se puede anidar con otro begin itemize para sublistas de objetos

% \newpage  Forzar nueva p�gina

% Las im�genes a�n no las entiendo porque es complicado, lo suyo es echarle un google

% Cualquier otra cosa que se necesite echandole 1 google en ingl�s siempre hay alguien que lo resuelve,
% est� muy bien documentado, yo ya te digo que no he necesitado aprender LaTeX como tal si no que me he
% puesto sobre la marcha buscando lo que necesitaba y aqu� estoy
